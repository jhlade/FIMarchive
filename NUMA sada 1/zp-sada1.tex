%----------------------------------------------------------------------------------------
%	PACKAGES AND OTHER DOCUMENT CONFIGURATIONS
%----------------------------------------------------------------------------------------

\documentclass[a4paper,10pt]{article} 

\usepackage[czech]{babel} % čeština
\usepackage{tabularx}
\usepackage{footnote}

\usepackage{amsmath}
\usepackage{mathtools}
%%\usepackage{matrix}

\usepackage{fontspec} % fonty
\defaultfontfeatures{Mapping=tex-text}
%\setmainfont{Calibri-Light} % Main document font
%\setmainfont{Lucida Grande}

\usepackage{xunicode,xltxtra,url,parskip} % Formatting packages

\usepackage{fancyhdr}

%\usepackage[usenames,dvipsnames]{xcolor} % Required for specifying custom colors

\usepackage[big]{layaureo} % Margin formatting of the A4 page, an alternative to layaureo
                           % can be \usepackage{fullpage} To reduce the height of the top
                           % margin uncomment:
                           % \addtolength{\voffset}{-1.3cm}
\geometry{left=15mm,right=15mm}

\usepackage{hyperref} % Required for adding links	and customizing them
%\definecolor{linkcolour}{rgb}{0,0.5,0.3} % Link color
%\hypersetup{colorlinks,breaklinks,urlcolor=linkcolour,linkcolor=linkcolour} % Set link colors throughout the document

\usepackage{titlesec} % Used to customize the \section command
\titleformat{\section}{\bf\raggedright}{\thesection.~}{0em}{}
\titlespacing{\section}{0pt}{3pt}{3pt} % Spacing around sections
\titleformat{\subsection}{\bf\raggedright}{\thesubsection~}{0em}{}

% PDF settings
\author{Bc. Jan Hladěna}
\hypersetup
{
    pdfauthor={Bc. Jan Hladěna},
    pdfsubject={KNUMA - Zápočtový úkol, sada 1}, 
    pdftitle={KNUMA - Zápočtový úkol, sada 1}
}

\lhead{\textsc{K-AI2-1 KNUMA - Zápočtový úkol, sada 1 - LS 2015/16}}
\rhead{\href{mailto:jan.hladena@uhk.cz}{\textsc{Jan Hladěna}, I1500705}}
\cfoot{}
\renewcommand{\headrulewidth}{0.4pt}
%\renewcommand{\footrulewidth}{0.4pt}

\def\doubleunderline#1{\underline{\underline{#1}}}

\begin{document}
\pagestyle{fancy}

%\font\fb=''[cmr10]'' % Change the font of the \LaTeX command under the skills section

%----------------------------------------------------------------------------------------
% ZÁPOČET
%----------------------------------------------------------------------------------------

\section{Úlohy na~absolutní a~relativní chybu výpočtu}

\par c) Určete výkon $P~=~U{\cdot}I$ a~elektrický odpor
$R~=~\frac{U}{I}$ topné spirály, jestliže bylo naměřeno napětí $U~=~(230~{\pm}~5)~V$
a~elektrický proud $I~=~(5~{\pm}~0,4)~A$. V~obou případech určete také relativní chybu.

\par {~}

%%%% Výpočet

\par Výkon P [W]

\begin{minipage}[t]{.6\textwidth}

$
\begin{array}{lcl}
P & = & U{\cdot}I~+~\epsilon{(P)}~=~U{\cdot}I~\pm~(|U|{\cdot}{\epsilon}(I)~+~|I|{\cdot}{\epsilon}(U)) \\
\\
P & = & 230~\cdot~5~\pm~(|230|{\cdot}0,4~+~|5|{\cdot}5) \\
  & = & 1150~\pm~(92~+~25) \\
  & = & \doubleunderline{1150~\pm~117~W}
\end{array}
$

\par {~}
\par {~}


\end{minipage}%
\begin{minipage}[t]{.4\textwidth}

$
\begin{array}{lcl}
\epsilon(U) & = & 5~V \\
\epsilon(I) & = & 0,4~A 
\end{array}
$

\par {~}

$
\begin{array}{lcl}
\delta{U} & = & \frac{5}{230}~\cdot~100~=~2,17~\% \\
\delta{I} & = & \frac{0,4}{5}~\cdot~100~=~8~\% 
\end{array}
$

\par {~}

\end{minipage}

\bigskip
\par{~}
\bigskip

\par Relativní chyba výkonu P

%% delta P
\begin{minipage}[t]{.6\textwidth}

$
\begin{array}{lcl}
\delta{P} & = & \frac{\epsilon{(P)}}{P}~\cdot~100 \\
\\
          & = & \frac{117}{1150}~\cdot~100 \\
\\
          & = & \doubleunderline{10,17~\%}
\end{array}
$

\end{minipage}%
\begin{minipage}[t]{.4\textwidth}

$
\begin{array}{lcl}
\delta{P} & = & \delta{U}~+~\delta{I} \\
\\
          & = & 2,17 + 8 \\
\\
          & = & \doubleunderline{10,17~\%}
\end{array}
$

\end{minipage}

\bigskip
\par{~}
\bigskip

\par Elektrický odpor R [$\Omega$]
%%% odpor

$
\begin{array}{lcl}
R & = & \frac{U}{I}~+~\epsilon{(R)} = \frac{U}{I}~\pm~(\frac{\epsilon{(U)}}{|I|}~+~\frac{|U|}{I^{2}}{\cdot}\epsilon{(I)}) \\
\\
  & = & \frac{U}{I}~\pm~\frac{|I|{\cdot}\epsilon{(U)}~+~|U|{\cdot}\epsilon{(I)}}{I^{2}} \\
\\
R & = & \frac{230}{5}~\pm~\frac{|5|{\cdot}{5}~+~|230|{\cdot}{0,4}}{5^{2}} \\
\\
  & = & 46~\pm~\frac{25~+~92}{25}~=46~\pm~\frac{117}{25} \\
\\
  & = & \doubleunderline{46~\pm~4,68~\Omega}
\end{array}
$

\bigskip
\par{~}
\bigskip

\par Relativní chyba odporu R

\begin{minipage}[t]{.6\textwidth}

$
\begin{array}{lcl}
\delta{R} & = & \frac{\epsilon{(R)}}{R}~\cdot~100 \\
\\
          & = & \frac{4,68}{46}~\cdot~100 \\
\\
          & = & \doubleunderline{10,17~\%}
\end{array}
$

\end{minipage}%
\begin{minipage}[t]{.4\textwidth}

$
\begin{array}{lcl}
\delta{R} & = & \delta{U}~+~\delta{I} \\
\\
          & = & 2,17 + 8 \\
\\
          & = & \doubleunderline{10,17~\%}
\end{array}
$

\end{minipage}

\newpage
\section{Pomocí Gaussovy eliminační metody a~Gaussovy eliminační metody s~částečnou pivotací
řešte soustavy lineárních algebraických rovnic}

$
a)~
\left( \begin{array}{rrrr|r}
1 & -1 & -1 &  2 &   6 \\
0 &  4 &  4 & -1 & -13 \\
2 &  2 &  1 & -3 &  -6 \\
1 &  0 & -2 &  1 &   4 
\end{array} \right)
$

\par {~}

%%%% Výpočet

\par Jsou použity funkce z~\path{N:\UKAZKY\Prazak\NUMA\B_cviceni\cv02\matlab}. \\


%% Zadání A %%%%%%%%%%%%%%%%%%%%%%%%%%%%%%%%%%%%%%%%%%%%%%%%%%%%%%%%%%%%%%%%%%%%%%%%%%%%%%

\par Zadání matice A:

\begin{verbatim}
>> A = [[1, -1, -1, 2, 6]; [0, 4, 4, -1, -13]; [2, 2, 1, -3, -6]; [1, 0, -2, 1, 4]]
\end{verbatim}


%% Výpis - zadání A %%%%%%%%%%%%%%%%%%%%%%%%%%%%%%%%%%%%%%%%%%%%%%%%%%%%%%%%%%%%%%%%%%%%%%
\begin{minipage}[t]{.5\textwidth}
\begin{verbatim}

A =

     1    -1    -1     2     6
     0     4     4    -1   -13
     2     2     1    -3    -6
     1     0    -2     1     4

\end{verbatim}
\end{minipage}%
\begin{minipage}[t]{.5\textwidth}

\bigskip
\bigskip

$
A~=~
\left( \begin{array}{rrrr|r}
1 & -1 & -1 &  2 &   6 \\
0 &  4 &  4 & -1 & -13 \\
2 &  2 &  1 & -3 &  -6 \\
1 &  0 & -2 &  1 &   4 
\end{array} \right)
$

\vfill

\end{minipage}

%% Čtvercová matice A %%%%%%%%%%%%%%%%%%%%%%%%%%%%%%%%%%%%%%%%%%%%%%%%%%%%%%%%%%%%%%%%%%%%

\par Čtvercová matice U:

\begin{verbatim}
>> U = A(1:4, 1:4)
\end{verbatim}

%% Výpis - čtvercová matice U %%%%%%%%%%%%%%%%%%%%%%%%%%%%%%%%%%%%%%%%%%%%%%%%%%%%%%%%%%%%

\begin{minipage}[t]{.5\textwidth}
\begin{verbatim}

U =

     1    -1    -1     2
     0     4     4    -1
     2     2     1    -3
     1     0    -2     1 

\end{verbatim}
\end{minipage}%
\begin{minipage}[t]{.5\textwidth}

\bigskip
\bigskip

$
U~=~
\left( \begin{array}{rrrr}
1 & -1 & -1 &  2 \\
0 &  4 &  4 & -1 \\
2 &  2 &  1 & -3 \\
1 &  0 & -2 &  1 
\end{array} \right)
$

\vfill
\end{minipage}

%% Vektor pravých stran %%%%%%%%%%%%%%%%%%%%%%%%%%%%%%%%%%%%%%%%%%%%%%%%%%%%%%%%%%%%%%%%%%

\par Vektor pravých stran:

\begin{verbatim}
>> y = A(1:4, 5)
\end{verbatim}

%% Výpis - vektor pravých stran %%%%%%%%%%%%%%%%%%%%%%%%%%%%%%%%%%%%%%%%%%%%%%%%%%%%%%%%%%
\begin{minipage}[t]{.5\textwidth}
\begin{verbatim}

y =

       6
     -13
      -6
       4

\end{verbatim}
\end{minipage}%
\begin{minipage}[t]{.5\textwidth}

\bigskip
\bigskip

$
y~=~
\left( \begin{array}{r}
  6 \\
-13 \\
 -6 \\
  4 
\end{array} \right)
$

\vfill
\end{minipage}

%% Eliminiace GEM %%%%%%%%%%%%%%%%%%%%%%%%%%%%%%%%%%%%%%%%%%%%%%%%%%%%%%%%%%%%%%%%%%%%%%%%

\newpage
\subsection{Eliminace GEM}

%% U1
\begin{verbatim}
>> [U1, y1] = eliminace1(U, y)
\end{verbatim}

\begin{minipage}[t]{.5\textwidth}
\begin{verbatim}

U1 =

    1.0000   -1.0000   -1.0000    2.0000
         0    4.0000    4.0000   -1.0000
         0         0   -1.0000   -6.0000
         0         0         0   11.2500

\end{verbatim}
\end{minipage}%
\begin{minipage}[t]{.5\textwidth}

\bigskip
\bigskip

$
U_1~=~
\left( \begin{array}{rrrr}
1 & -1 & -1 &  2 \\
0 &  4 &  4 & -1 \\
0 &  0 & -1 & -6 \\
0 &  0 &  0 & 11,25 
\end{array} \right)
$
\vfill
\end{minipage}

%% y1
\begin{minipage}[t]{.5\textwidth}
\begin{verbatim}

y1 =

    6.0000
  -13.0000
   -5.0000
   11.2500

\end{verbatim}
\end{minipage}%
\begin{minipage}[t]{.5\textwidth}

\bigskip
\bigskip

$
y_1~=~
\left( \begin{array}{r}
  6 \\
-13 \\
 -5 \\
 11,25
\end{array} \right)
$
\vfill
\end{minipage}

%% Zpětné %%%%%%%%%%%%%%%%%%%%%%%%%%%%%%%%%%%%%%%%%%%%%%%%%%%%%%%%%%%%%%%%%%%%%%%%%%%%%%%%

\par Zpětné dosazení:

\begin{verbatim}
>> [x1] = back(U1, y1)
\end{verbatim}

%% x1
\begin{minipage}[t]{.5\textwidth}
\begin{verbatim}

x1 =

     1
    -2
    -1
     1

\end{verbatim}
\end{minipage}%
\begin{minipage}[t]{.5\textwidth}

\bigskip
\bigskip

$
\overline{x_1}~=~
\left( \begin{array}{r}
  1 \\
 -2 \\
 -1 \\
  1
\end{array} \right)
$
\vfill
\end{minipage}


%% Eliminiace GEM s pivotací %%%%%%%%%%%%%%%%%%%%%%%%%%%%%%%%%%%%%%%%%%%%%%%%%%%%%%%%%%%%%

\subsection{Eliminace GEM s~částečnou pivotací}

%% U2
\begin{verbatim}
>> [U2, y2] = eliminace2(U, y)
\end{verbatim}

\begin{minipage}[t]{.5\textwidth}
\begin{verbatim}

U2 =

    2.0000    2.0000    1.0000   -3.0000
         0    4.0000    4.0000   -1.0000
         0         0   -1.5000    2.2500
         0         0         0    3.7500

\end{verbatim}
\end{minipage}%
\begin{minipage}[t]{.5\textwidth}

\bigskip
\bigskip

$
U_2~=~
\left( \begin{array}{rrrr}
2 &  2 &  1   & -3 \\
0 &  4 &  4   & -1 \\
0 &  0 & -1,5 &  2,25 \\
0 &  0 &  0   &  3,75 
\end{array} \right)
$
\vfill
\end{minipage}

%% y2
\begin{minipage}[t]{.5\textwidth}
\begin{verbatim}

y2 =

   -6.0000
  -13.0000
    3.7500
    3.7500

\end{verbatim}
\end{minipage}%
\begin{minipage}[t]{.5\textwidth}

\bigskip
\bigskip

$
y_2~=~
\left( \begin{array}{r}
 -6 \\
-13 \\
  3,75 \\
  3,75
\end{array} \right)
$
\vfill
\end{minipage}

%% Zpětné %%%%%%%%%%%%%%%%%%%%%%%%%%%%%%%%%%%%%%%%%%%%%%%%%%%%%%%%%%%%%%%%%%%%%%%%%%%%%%%%
\newpage
\par Zpětné dosazení:

\begin{verbatim}
>> [x2] = back(U2, y2)
\end{verbatim}

%% x2
\begin{minipage}[t]{.5\textwidth}
\begin{verbatim}

x2 =

     1
    -2
    -1
     1

\end{verbatim}
\end{minipage}%
\begin{minipage}[t]{.5\textwidth}

\bigskip
\bigskip

$
\overline{x_2}~=~
\left( \begin{array}{r}
  1 \\
 -2 \\
 -1 \\
  1
\end{array} \right)
$
\vfill
\end{minipage}



\newpage
\section{Pomocí Jacobiovy nebo Gauss-Seidelovou metodou řešte soustavy lineárních algebraických
rovnic. Ověřujte podmínky konvergence a~soustavu případně upravte tak, aby byla použitá metoda
konvergentní.}

$
b)~
\left( \begin{array}{rrr|r}
1 & 5 & 7 & -2 \\
9 & 5 & 3 &  2 \\
2 & 7 & 4 & -8
\end{array} \right)
$

\par {~}

%%%% Výpočet %%%%%%%%%%%%%%%%%%%%%%%%%%%%%%%%%%%%%%%%%%%%%%%%%%%%%%%%%%%%%%%%%%%%%%%%%%%%%

\par Jsou použity funkce z~\path{N:\UKAZKY\Prazak\NUMA\B_cviceni\cv04\matlab}. \\


\par Zadání matice A:

\begin{verbatim}
>> A = [[1, 5, 7, -2]; [9, 5, 3, 2]; [2, 7, 4, -8]]
\end{verbatim}


%% Výpis - zadání A %%%%%%%%%%%%%%%%%%%%%%%%%%%%%%%%%%%%%%%%%%%%%%%%%%%%%%%%%%%%%%%%%%%%%%
\begin{minipage}[t]{.5\textwidth}
\begin{verbatim}

A =

     1     5     7    -2
     9     5     3     2
     2     7     4    -8

\end{verbatim}
\end{minipage}%
\begin{minipage}[t]{.5\textwidth}

\bigskip
\bigskip

$
A~=~
\left( \begin{array}{rrr|r}
1 & 5 & 7 & -2 \\
9 & 5 & 3 &  2 \\
2 & 7 & 4 & -8
\end{array} \right)
$

\vfill

\end{minipage}

\par Byla zvolena Gauss-Seidelova metoda. Čtvercová matice $U$ je regulární, nesymetrická,
v~tomto tvaru není diagonálně dominantní. Záměnou řádků 1~a~2 a~následnou záměnou řádků
2~a~3 původní matice $A$ se matice $U$ stává ostře řádkově diagonálně dominantní. Tato
vlastnost je postačující pro~splnění podmínek konvergence iterační metody. \\

\begin{verbatim}
>> A([ 1 2 ], :) = A([ 2 1 ], :)
>> A([ 2 3 ], :) = A([ 3 2 ], :)
\end{verbatim}

%% Výpis - záměna A %%%%%%%%%%%%%%%%%%%%%%%%%%%%%%%%%%%%%%%%%%%%%%%%%%%%%%%%%%%%%%%%%%%%%%
\begin{minipage}[t]{.5\textwidth}
\begin{verbatim}

A =

     9     5     3     2
     2     7     4    -8
     1     5     7    -2

\end{verbatim}
\end{minipage}%
\begin{minipage}[t]{.5\textwidth}

\bigskip
\bigskip

$
A~=~
\left( \begin{array}{rrr|r}
9 & 5 & 3 &  2 \\
2 & 7 & 4 & -8 \\
1 & 5 & 7 & -2 
\end{array} \right)
$

\vfill
\end{minipage}

%% Matice U %%%%%%%%%%%%%%%%%%%%%%%%%%%%%%%%%%%%%%%%%%%%%%%%%%%%%%%%%%%%%%%%%%%%%%%%%%%%%%
\par Čtvercová matice:

\begin{verbatim}
>> U = A([1:3],[1:3])
\end{verbatim}

%% Výpis - U %%%%%%%%%%%%%%%%%%%%%%%%%%%%%%%%%%%%%%%%%%%%%%%%%%%%%%%%%%%%%%%%%%%%%%%%%%%%%
\begin{minipage}[t]{.5\textwidth}
\begin{verbatim}

U =

     9     5     3 
     2     7     4 
     1     5     7 

\end{verbatim}
\end{minipage}%
\begin{minipage}[t]{.5\textwidth}

\bigskip
\bigskip

$
U~=~
\left( \begin{array}{rrr}
9 & 5 & 3 \\
2 & 7 & 4 \\
1 & 5 & 7 
\end{array} \right)
$

\vfill
\end{minipage}

%% Vektor b %%%%%%%%%%%%%%%%%%%%%%%%%%%%%%%%%%%%%%%%%%%%%%%%%%%%%%%%%%%%%%%%%%%%%%%%%%%%%%
\par Vektor pravých stran:

\begin{verbatim}
>> b = A([1:3], 4)
\end{verbatim}


%% Vektor b %%%%%%%%%%%%%%%%%%%%%%%%%%%%%%%%%%%%%%%%%%%%%%%%%%%%%%%%%%%%%%%%%%%%%%%%%%%%%%
\begin{minipage}[t]{.5\textwidth}
\begin{verbatim}

b =

     2
    -8
    -2

\end{verbatim}
\end{minipage}%
\begin{minipage}[t]{.5\textwidth}

\bigskip
\bigskip

$
b~=~
\left( \begin{array}{r}
  2 \\
 -8 \\
 -2
\end{array} \right)
$
\vfill
\end{minipage}

\newpage

\par Použití Gauss-Seidelovy iterační metody s~maximálním počtem 50~iterací a~prázdným
počátečním výsledkem:

\begin{verbatim}
>> x0 = zeros(3, 1)
>> x = gauss_seidel(U, b, x0, 50)
\end{verbatim}

%% výsledek x %%%%%%%%%%%%%%%%%%%%%%%%%%%%%%%%%%%%%%%%%%%%%%%%%%%%%%%%%%%%%%%%%%%%%%%%%%%%%%
\begin{minipage}[t]{.5\textwidth}
\begin{verbatim}

x =

     1
    -2
     1

\end{verbatim}
\end{minipage}%
\begin{minipage}[t]{.5\textwidth}

\bigskip
\bigskip

$
\overline{x}~=~
\left( \begin{array}{r}
  1 \\
 -2 \\
  1
\end{array} \right)
$
\vfill
\end{minipage}



\newpage
\section{Použijte \emph{LU} rozklad matice a~nalezněte determinant matice i~matici inverzní.}

$
c)~
\left( \begin{array}{rrr}
-8 &  1 & -2 \\
 2 & -6 & -1 \\
-3 & -1 &  7 
\end{array} \right)
$

\par {~}

%%%% Výpočet %%%%%%%%%%%%%%%%%%%%%%%%%%%%%%%%%%%%%%%%%%%%%%%%%%%%%%%%%%%%%%%%%%%%%%%%%%%%%

\par Jsou použity funkce z~\path{N:\UKAZKY\Prazak\NUMA\B_cviceni\cv03\matlab\LU}. \\


%% Zadání A %%%%%%%%%%%%%%%%%%%%%%%%%%%%%%%%%%%%%%%%%%%%%%%%%%%%%%%%%%%%%%%%%%%%%%%%%%%%%%

\par Zadání matice A:

\begin{verbatim}
>> A = [[-8, 1, -2]; [2, -6, -1]; [-3, -1, 7]]
\end{verbatim}


%% Výpis - zadání A %%%%%%%%%%%%%%%%%%%%%%%%%%%%%%%%%%%%%%%%%%%%%%%%%%%%%%%%%%%%%%%%%%%%%%
\begin{minipage}[t]{.5\textwidth}
\begin{verbatim}

A =

    -8     1    -2
     2    -6    -1
    -3    -1     7

\end{verbatim}
\end{minipage}%
\begin{minipage}[t]{.5\textwidth}

\bigskip

$
A~=~
\left( \begin{array}{rrr}
-8 &  1 & -2 \\
 2 & -6 & -1 \\
-3 & -1 &  7 
\end{array} \right)
$

\vfill

\end{minipage}


%% Dotaz na LU rozklad %%%%%%%%%%%%%%%%%%%%%%%%%%%%%%%%%%%%%%%%%%%%%%%%%%%%%%%%%%%%%%%%%%%

\par Provedení LU rozkladu matice A:

\begin{verbatim}

>> [L, U] = LU_rozklad(A)

\end{verbatim}

%%% L %%%%%%%%%%%%%%%%%%%%%%%%%%%%%%%%%%%%%%%%%%%%%%%%%%%%%%%%%%%%%%%%%%%%%%%%%%%%%%%%%%%%
\begin{minipage}[t]{.5\textwidth}
\begin{verbatim}
L =

    1.0000         0         0
   -0.2500    1.0000         0
    0.3750    0.2391    1.0000

\end{verbatim}
\end{minipage}%
\begin{minipage}[t]{.5\textwidth}

\bigskip

$
A_L~=~
\left( \begin{array}{rrr}
  1 &  0 & 0 \\
 -0,25 & 1 & 0 \\
  0,375 & 0,2391 &  1 
\end{array} \right)
$

\vfill

\end{minipage}

%%% U %%%%%%%%%%%%%%%%%%%%%%%%%%%%%%%%%%%%%%%%%%%%%%%%%%%%%%%%%%%%%%%%%%%%%%%%%%%%%%%%%%%%
\begin{minipage}[t]{.5\textwidth}
\begin{verbatim}
U =

   -8.0000    1.0000   -2.0000
         0   -5.7500   -1.5000
         0         0    8.1087

\end{verbatim}
\end{minipage}%
\begin{minipage}[t]{.5\textwidth}

\bigskip

$
A_U~=~
\left( \begin{array}{rrr}
  -8 &  1 	& -2 \\
   0 & -5,75 & -1,5 \\
   0 & 0	&  8,1087 
\end{array} \right)
$

\vfill

\end{minipage}

%% dotaz na determinant A %%%%%%%%%%%%%%%%%%%%%%%%%%%%%%%%%%%%%%%%%%%%%%%%%%%%%%%%%%%%%%%%

\par Výpočet determinantu matice A:

\begin{verbatim}

>> detA = determinant(A)

\end{verbatim}

%% výpis determinant A %%%%%%%%%%%%%%%%%%%%%%%%%%%%%%%%%%%%%%%%%%%%%%%%%%%%%%%%%%%%%%%%%%%
\begin{minipage}[t]{.5\textwidth}
\begin{verbatim}

detA =

  373.0000
  
\end{verbatim}
\end{minipage}%
\begin{minipage}[t]{.5\textwidth}

\bigskip

$
\det~A~=~
\left| \begin{array}{rrr}
  -8 &  1 	& -2 \\
   0 & -5,75 & -1,5 \\
   0 & 0	&  8,1087 
\end{array} \right|
~=~373
$

\vfill

\end{minipage}

%% inverzní matice %%%%%%%%%%%%%%%%%%%%%%%%%%%%%%%%%%%%%%%%%%%%%%%%%%%%%%%%%%%%%%%%%%%%%%%

\newpage
\par Inverze matice A:

\begin{verbatim}

>> invA = inv(A)

\end{verbatim}

%% výpis inverzní A %%%%%%%%%%%%%%%%%%%%%%%%%%%%%%%%%%%%%%%%%%%%%%%%%%%%%%%%%%%%%%%%%%%%%%
\begin{minipage}[t]{.5\textwidth}
\begin{verbatim}

invA =

   -0.1153   -0.0134   -0.0349
   -0.0295   -0.1662   -0.0322
   -0.0536   -0.0295    0.1233
  
\end{verbatim}
\end{minipage}%
\begin{minipage}[t]{.5\textwidth}

\bigskip
\bigskip

$
A^{-1}~=~
\left( \begin{array}{rrr}
   -0,1153  & -0,0134  & -0,0349 \\
   -0,0295  & -0,1662  & -0,0322 \\
   -0,0536  & -0,0295  &  0,1233
\end{array} \right)
$

\vfill

\end{minipage}

%%%%%%%%%%%%%%%%%%%%%%%%%%%%%%%%%%%%%%%%%%%%%%%%%%%%%%%%%%%%%%%%%%%%%%%%%%%%%%%%%%%%%%%%%%


%----------------------------------------------------------------------------------------
% KONEC
%----------------------------------------------------------------------------------------

\end{document}
