%----------------------------------------------------------------------------------------
%	PACKAGES AND OTHER DOCUMENT CONFIGURATIONS
%----------------------------------------------------------------------------------------

\documentclass[a4paper,10pt]{article} 

\usepackage[czech]{babel} % čeština
\usepackage{tabularx}
\usepackage{footnote}

\usepackage{amsmath}
\usepackage{mathtools}
%%\usepackage{matrix}

\usepackage{upgreek} %% stojatá řecká písmena!

\usepackage{fontspec} % fonty
\defaultfontfeatures{Mapping=tex-text}
%\setmainfont{Calibri-Light} % Main document font
%\setmainfont{Lucida Grande}

\usepackage{xunicode,xltxtra,url,parskip} % Formatting packages

\usepackage{fancyhdr}

%\usepackage[usenames,dvipsnames]{xcolor} % Required for specifying custom colors

\usepackage[big]{layaureo} % Margin formatting of the A4 page, an alternative to layaureo
                           % can be \usepackage{fullpage} To reduce the height of the top
                           % margin uncomment:
                           % \addtolength{\voffset}{-1.3cm}
\geometry{left=15mm,right=15mm}

\usepackage{hyperref} % Required for adding links	and customizing them
%\definecolor{linkcolour}{rgb}{0,0.5,0.3} % Link color
%\hypersetup{colorlinks,breaklinks,urlcolor=linkcolour,linkcolor=linkcolour} % Set link colors throughout the document

\usepackage{titlesec} % Used to customize the \section command
\titleformat{\section}{\bf\raggedright}{\thesection.~}{0em}{}
\titlespacing{\section}{0pt}{3pt}{3pt} % Spacing around sections
\titleformat{\subsection}{\bf\raggedright}{\thesubsection~}{0em}{}

% PDF settings
\author{Bc. Jan Hladěna}
\hypersetup
{
    pdfauthor={Bc. Jan Hladěna},
    pdfsubject={KNUMA - Zápočtový úkol, sada 3}, 
    pdftitle={KNUMA - Zápočtový úkol, sada 3}
}

\lhead{\textsc{K-AI2-1 KNUMA - Zápočtový úkol, sada 3 - LS 2015/16}}
\rhead{\href{mailto:jan.hladena@uhk.cz}{\textsc{Jan Hladěna}, I1500705}}
\cfoot{}
\renewcommand{\headrulewidth}{0.4pt}
%\renewcommand{\footrulewidth}{0.4pt}

\def\doubleunderline#1{\underline{\underline{#1}}}
\def\d#1{\,\mathop{\mathrm{d}#1}}

\usepackage{chngcntr}

\counterwithin*{equation}{section}
\counterwithin*{equation}{subsection}

\begin{document}
\pagestyle{fancy}

%\font\fb=''[cmr10]'' % Change the font of the \LaTeX command under the skills section

%----------------------------------------------------------------------------------------
% ZÁPOČET
%----------------------------------------------------------------------------------------

\section{Řešte úlohy týkající se interpolace.}
\label{sec:lagrange}
\par Zvoleno zadání a) Napište Lagrangerův interpolační polynom $L(x)$, který interpoluje
hodnoty

\begin{center}
	\begin{tabular}{|c|c|c|c|}
		\hline
		$x$ & 1,2 & 1,8 & 2,5 \\
		\hline
		$f(x)$ & 2,847 & 1,680 & 0,039 \\
		\hline
	\end{tabular}
\end{center}

a~použijte ho pro~odhad $f(1,5)$ a~$f(2,0)$. 

%%%% Výpočet

\par Lagrangerova metoda konstrukce interpolačního polynomu sestává z~dílčích konstrukcí
fundamentálních polynomů $n$-tého stupňě $l_i(x), i=0,\dots,n$, pro~které platí

\[l_i(x_j)= 
\begin{dcases}
    {~}0, & i\neq{j},\\
    {~}1, & i=j.\\
\end{dcases}
\]

\par Kořeny každého takového fundamentálního polynomu $l_i(x)$ jsou pak
$x_0, x_1, \dots, x_{n-1}, x_n$, tedy platí

\begin{equation}
l_i(x)=C_i\cdot(x-x_0)(x-x_1)\dots(x-x_{i-1})(x-x_{i+1})\dots(x-x_{n-1})(x-x_n),
\end{equation}

přičemž podmínka $l_i(x_i)=1$ bude splněna, pokud

\begin{equation}
C_i=\dfrac{1}{(x_i-x_0)(x_i-x_1)\dots(x-x_{i-1})(x-x_{i+1})\dots(x_i-x_{n-1})(x_i-x_n)}.
\end{equation}

\par Potom je fundamentální polynom $l_i(x)$

\begin{equation}
\label{eq:li}
l_i(x)=\dfrac{(x-x_0)(x-x_1)\dots(x-x_{i-1})(x-x_{i+1})\dots(x-x_{n-1})(x-x_n)}
{(x_i-x_0)(x_i-x_1)\dots(x-x_{i-1})(x-x_{i+1})\dots(x_i-x_{n-1})(x_i-x_n)}.
\end{equation}

\par Tento polynom nabývá nulové hodnoty ve~všech uzlech, kromě jediného, ve~kterém
nabývá hodnoty 1. Celková hodnota lineární kombinace fundamentálních polynomů v~$i$-tém
uzlu je určena pouze odpovídajícím $i$-tým fundamentálním polynomem $l_i(x)$, zbylé
polynomy výsledek neovlivní. Lagrangerův interpolační polynom $L_n(x)$ má potom tvar

\begin{equation}
\label{eq:Ln}
L_n(x)=\sum_{i=0}^{n}y_{i}l_{i}(x).
\end{equation}

\subsection{Konstrukce Lagrangerova polynomu}
\par Vytvoření odpovídajících fundamentálních polynomů ze~zadané tabulky lze realizovat
dosazením do~rovnice (\ref{eq:li}):

\[\arraycolsep=1.4pt\def\arraystretch{2.4}
\begin{array}{cclcl}
l_0(x)&=&\dfrac{(x-1,8)(x-2,5)}{(1,2-1,8)(1,2-2,5)}&=&\dfrac{5}{39}\cdot(10x^2-43x+45)\\
l_1(x)&=&\dfrac{(x-1,2)(x-2,5)}{(1,8-1,2)(1,8-2,5)}&=&-\dfrac{50}{21}\cdot(x^2-3,7x+3)\\
l_2(x)&=&\dfrac{(x-1,2)(x-1,8)}{(2,5-1,2)(2,5-1,8)}&=&\dfrac{4}{91}\cdot(25x^2-75x+54)\\
\end{array}
\]

\newpage
\par a~jejich následné dosazení do~vztahu (\ref{eq:Ln}) vede k~získání výsledného
Lagrangerova polynomu stupně $n=2$:

\[
L_2=2,847\cdot{l_0(x)}+1,680\cdot{l_1(x)}+0,039\cdot{l_2(x)}
\]

\[
L_2 = 2,847\cdot\dfrac{5\cdot(10x^2-43x+45)}{39}
+1,680\cdot\dfrac{-50\cdot(x^2-3,7x+3)}{21}
+0,039\cdot\dfrac{4\cdot(25x^2-75x+54)}{91}
\]\\

\par Hledaný Lagrangerův interpolační polynom je
\doubleunderline{$L_2(x)=-0,307143x^2-1,02357x+4,51757$}. \\

\subsection{Odhad podle získaného Lagrangerova polynomu}
\par Výpočet $L_2(x)$ pro~zadané hodnoty $f(1,5)$ a~$f(2,0)$: \\

\par $L_2(1,5) = -0,307143\cdot(1,5)^2-1,02357\cdot(1,5)+4,51757 =$
\doubleunderline{$2,29114$}

\par  $L_2(2,0) = -0,307143\cdot(2)^2-1,02357\cdot(2)+4,51757 =$
\doubleunderline{$1,24186$} \\

\par Ověření v~prostředí \texttt{MATLAB} s~využitím funkce \texttt{lagrange.m}
z~\path{N:\UKAZKY\Prazak\NUMA\B_cviceni\cv05\matlab}:

\begin{verbatim}
>> x = [ 1.2 1.8 2.5 ];
>> y = [ 2.847 1.680 0.039 ];
>> u = [ 1.5 2.0 ];
>> v = lagrange(x, y, u)

v =
    2.2911    1.2419
\end{verbatim}

%----------------------------------------------------------------------------------------
\newpage
\section{Řešte úlohy týkající se numerické integrace.}

\par Zvoleno zadání c) Pomocí lichoběžníkového i~Simpsonova pravidla určete aproximaci
určitého integrálu

$$\int_{0}^{1}\sqrt{\cos{x}}\d{x}$$

\par Použijte $n=10$ a~pro~lichoběžníkové pravidlo odhadněte chybu. \\

%%%% Výpočet

\par Pomocné derivace: \\

$f'(x)=\dfrac{1}{2}\cdot\dfrac{1}{\sqrt{\cos{x}}}\cdot({-\sin{x}})=
-\dfrac{\sin{x}}{2\cdot\sqrt{\cos{x}}}$

$f''(x)=-\dfrac{1}{2}\cdot\left(\dfrac{\sin{x}}
{\sqrt{\cos{x}}}\right)'=-\dfrac{1}{2}\cdot\dfrac{\sqrt{\cos{x}}(\sin{x})'
-\sin{x}(\sqrt{\cos{x}})'}{\cos{x}}$

$f''(x)=-\dfrac{1}{2}\cdot\dfrac{\sqrt{\cos{x}}\cdot(-\sin{x})
-\left(\dfrac{(\cos{x})'}{2\cdot\sqrt{\cos{x}}}\right)}{\cos{x}}=
\dfrac{1}{2}\cdot\dfrac{\dfrac{\sin{x}\cdot(-\sin{x})}
{2\cdot\sqrt{\cos{x}}}-{(\cos{x})}^{\frac{3}{2}}}{\cos{x}}=
\dfrac{\sin^{2}{x}-2}{4\cdot\cos^{\frac{3}{2}}{x}}$

\subsection{Lichoběžníkové pravidlo}

\par Lichoběžníkové pravidlo nahrazuje na~každém podintervalu $\langle x_{i-1}, x\rangle$
integrand $f(x)$ Lagrangerovým polynomem tak, jak bylo popsáno v~příkladě
\ref{sec:lagrange}. Při~$h=x_i-x_{x-1}$ pak pro~každý subinterval platí

\begin{equation}
\int_{x_{i-1}}^{x_i}f(x)\d{x}\,\approx\,\int_{x_{i-1}}^{x_i}L_{i,1}(x)\d{x},
\end{equation}

po~patřičné úpravě

\begin{equation}
\int_{x_{i-1}}^{x_i}f(x)\d{x}\,\approx\,\dfrac{h}{2}\left( f(x_{i-1})+f(x)\right).
\end{equation}

Pro~celý interval $\langle a, b \rangle$ lze aproximaci vyjádřit ve~tvaru

\begin{equation}
\int_{a}^{b}f(x)\d{x}\,=\,\sum_{i=1}^{n}\int_{x_{i-1}}^{x_i}f(x)\d{x}
\,\approx\,\sum_{i=1}^{n}\int_{x_{i-1}}^{x_i}L_{i,1}(x)\d{x},
\end{equation}

\begin{equation}
\label{eq:integral}
\int_{a}^{b}f(x)\d{x}\,\approx\,\dfrac{h}{2}\left[
f(a)+2\cdot\sum_{i=2}^{n-1}f(x_i)+f(b)
\right]
\end{equation}

%% chyba lichoběžníkového pravidla

\par Obecná chyba takto získané aproximace je rovna

\begin{equation}
R_{i,1}(f)=\dfrac{f''(\eta)}{2} \int_{x_{i-1}}^{x_i}(x-x_{i-1})(x-x_i)\d{x}
\end{equation}

pro~určitá $\eta_i\in\langle x_{x-1}, x_i \rangle$. Pro~celkový odhad chyby lze použít
vztah

\begin{equation}
\label{eq:err}
|\bar{R}_1(f)|\leq{n}\dfrac{h^3}{12}\bar{M}_2\,=\,\dfrac{\bar{M}_2}{12n^2}(b-a)^3.
\end{equation}

\newpage
\subsubsection{Aproximace určitého integrálu lichoběžníkovou metodou}
\par Pro~provedení výpočtu je třeba sestavit tabulku uzlů a~funkčních hodnot v~těchto
uzlech. Uvažujeme-li lichoběžníkovou metodu jako ekvidistantní kvadraturní formuli, potom
konstantní krok $h=\dfrac{b-a}{n}$ bude při~$n=10$ na~intervalu $\langle 0, 1 \rangle$
$h=\dfrac{1-0}{10}=0,1$.

\begin{center}
	\def\arraystretch{1.5}
	\begin{tabular}{|c||c|c|c|c|}
		\hline
			$i$ & $x_i$ & $f(x)=\sqrt{\cos{x}}$ & $f(x_i)$ & $f''(x_i)$ \\
			\hline\hline
			$0$ & $0$ & $\sqrt{\cos{0}}$ & $1$ & $-0,5$ \\
			\hline
			$1$ & $0,1$ & $\sqrt{\cos{0,1}}$ & $0,997499$ & $-0,5012599$ \\
			\hline
			$2$ & $0,2$ & $\sqrt{\cos{0,2}}$ & $0,9899831$ & $-0,5051615$ \\
			\hline
			$3$ & $0,3$ & $\sqrt{\cos{0,3}}$ & $0,9774132$ & $-0,5120885$ \\
			\hline
			$4$ & $0,4$ & $\sqrt{\cos{0,4}}$ & $0,9597192$ & $-0,5227481$ \\
			\hline
			$5$ & $0,5$ & $\sqrt{\cos{0,5}}$ & $0,9367938$ & $-0,5382926$ \\
			\hline
			$6$ & $0,6$ & $\sqrt{\cos{0,6}}$ & $0,9084798$ & $-0,5605419$ \\
			\hline
			$7$ & $0,7$ & $\sqrt{\cos{0,7}}$ & $0,8745526$ & $-0,592389$ \\
			\hline
			$8$ & $0,8$ & $\sqrt{\cos{0,8}}$ & $0,8346896$ & $-0,63857$ \\
			\hline
			$9$ & $0,9$ & $\sqrt{\cos{0,9}}$ & $0,7884225$ & $-0,7072147$ \\
			\hline
			$10$ & $1$ & $\sqrt{\cos{1}}$ & $0,7350526$ & $-0,8132473^{*}$ \\
			
		\hline
	\end{tabular}
\end{center}

\par Získané výsledky je možné dosadit do~vztahu (\ref{eq:integral}):

\[
\int_{0}^{1}\sqrt{\cos{x}}\d{x}\,\approx\,\dfrac{0,1}{2}\left[
f(0)+2\cdot\sum_{i=2}^{9}f(x_i)+f(10)
\right]
\]

\[
=0,05\cdot\left[
1+
2\cdot(
0,997499+0,9899831+\dots+0,8346896+0,7884225
)
+0,7350526
\right]=\doubleunderline{0,9135079 \pm R_1(f)}
\]

\par Ověření v~prostředí \texttt{R}:

\begin{verbatim}
> x <- seq(0, 1, 0.1)
> fx <- sqrt(cos(x))
> 0.1/2*(fx[1]+2*sum(fx[c(2:10)])+fx[11])
[1] 0.9135079
\end{verbatim}

\subsubsection{Odhah chyby u~lichoběžníkové metody}
\par Odhad chyby dosazením do~vztahu (\ref{eq:err}):

\[
R_1(f)\,=\,\dfrac{\max(|f''(x_i)|)}{12\cdot10^2}(1-0)^3\,=\,\dfrac{0,8132473}{120}\,
=\,\doubleunderline{\pm0,006777061}
\]

\par Ověření v~prostředí \texttt{R}:

\begin{verbatim}
> M2 <- max(abs(((sin(x))^2-2)/(4*(cos(x))^(3/2))))
> R1 <- (M2/(12*10^2))*(1-0)^3
> R1
[1] 0.0006777061
\end{verbatim}

\newpage
\subsection{Simpsonovo pravidlo}

\par Simpsonova metoda pracuje s~parabolami, spočívá tedy v~nahrazení integrandu $f(x)$
v~každém svém subintervalu polynomem druhého stupně. K~sestavení každého takového polynomu
je třeba znát právě tři body. Jako třetí bod je obyčejně použit střed každého
subintervalu.

\begin{equation}
\int_{i}^{j}f(x)\d{x}\,\approx\,\int_{i}^{j}L_2(x)\d{x}
\end{equation}

\par Při~označení $i=s_0$, $\dfrac{j-i}{2}=s_1$, $j=s_2$ lze získat tvar

\begin{equation}
\int_{i}^{j}L_2(x)\d{x}\,=\,
\int_{i}^{j}\left[
f(s_0)\dfrac{(x-s_1)(x-s_2)}{(s_0-s_1)(s_0-s_2)}+
f(s_1)\dfrac{(x-s_0)(x-s_2)}{(s_1-s_0)(s_1-s_2)}+
f(s_2)\dfrac{(x-s_0)(x-s_1)}{(s_2-s_0)(s_2-s_1)}
\right]\d{x}
\end{equation}

\par A~po~odpovídajících úpravách při~$h=s_1-s_0$

\begin{equation}
\int_{i}^{j}f(x)\d{x}\,\approx\,
\dfrac{h}{3}\left[
f(s_0)+4\cdot{f}(s_1)+f(s_2)
\right]
\end{equation}

\par Z~uvedeného vztahu vyplývá, že interval $\langle a, b \rangle$ složeného pravidla
musí být rozdělen na~$2n$ stejných částí s~podintervaly definovanými jako $\langle
x_{2i-2}, x_{2i} \rangle$ se~středem $s_i=x_{2i-1}$ pro~$i=1,\dots,n$, přičemž
$h=\dfrac{b-a}{2n}$.

\begin{equation}
\int_{a}^{b}f(x)\d{x}\,\approx\,
\sum_{i=1}^{n}\int_{x_{2i-2}}^{x_{2i}}L_{i,2}(x)\d{x}\,=\,
\dfrac{h}{3}\left[
\sum_{i=1}^{n}\left(
f(x_{2i-2})+4f(x_{2i-1})+f(x_{2i})
\right)
\right]
\end{equation}

\begin{equation}
\label{eq:simpson}
\int_{a}^{b}f(x)\d{x}\,\approx\,
\dfrac{h}{3}\left[
f(x_0)+2\sum_{i=1}^{n-1}f(x_{2i})+
4\sum_{i=1}^{n}f(x_{2i-1})+
f(x_{2n})
\right]
\end{equation}

\subsubsection{Aproximace určitého integrálu Simpsonovým pravidlem}

\par Určení kroku $h=\dfrac{b-a}{2n}$ pro~$n=10$: $h=\dfrac{1-0}{2\cdot{10}}=0,05$.

\begin{center}
	\def\arraystretch{1.5}
	\begin{tabular}{|c|c||c|c|}
		\hline
			$x_{2i}$ & $f(x_{2i})=\sqrt{\cos{x_{2i}}}$ & $x_{2i-1}$ & $f(x_{2i-1})=\sqrt{\cos{x_{2i-1}}}$ \\
			\hline
			\hline
			$x_0=0$ & $1$ & $x_1=0,05$ & $0,9993749$ \\
			\hline
			$x_2=0,1$ & $0,9974990$ & $x_3=0,15$ & $0,9943697$ \\
			\hline
			$x_4=0,2$ & $0,9899831$ & $x_5=0,25$ & $0,9843335$ \\
			\hline
			$x_6=0,3$ & $0.9774132$ & $x_7=0,35$ & $0,9692124$ \\
			\hline
			$x_8=0,4$ & $0,9597192$ & $x_9=0,45$ & $0,9489189$ \\
			\hline
			$x_{10}=0,5$ & $0,9367938$ & $x_{11}=0,55$ & $0,9233225$ \\
			\hline
			$x_{12}=0,6$ & $0,9084798$ & $x_{13}=0,65$ & $0,8922353$ \\
			\hline
			$x_{14}=0,7$ & $0,8745526$ & $x_{15}=0,75$ & $0,8553881$ \\
			\hline
			$x_{16}=0,8$ & $0,8346896$ & $x_{17}=0,85$ & $0,8123935$ \\
			\hline
			$x_{18}=0,9$ & $0,7884225$ & $x_{19}=0,95$ & $0,7626815$ \\
			\hline
			$x_{20}=1$ & $0,7350526$ & ~ & ~ \\
		\hline
	\end{tabular}
\end{center}

\newpage

\par Po~dosazení hodnot do~vztahu (\ref{eq:simpson}):

\[
\int_{0}^{1}f(x)\d{x}\,\approx\,
\dfrac{0,05}{3}\left[
1+2\sum_{i=1}^{9}f(x_{2i})+
4\sum_{i=1}^{10}f(x_{2i-1})+
0,7350526
\right]
\]

\[
\int_{0}^{1}f(x)\d{x}\,\approx\,
\dfrac{0,05}{3}\left[
1+2\cdot{8,267553}+
4\cdot{9,14223}+
0,7350526
\right]=\doubleunderline{0,9139847}
\]\\


\par Ověření v~prostředí \texttt{R}:

\begin{verbatim}
> x <- seq(0, 1, 0.05)
> fx <- sqrt(cos(x))
> (0.05/3)*(fx[1]+2*sum(fx[seq(3,19,2)])+4*sum(fx[seq(2,20,2)])+fx[21])
[1] 0.9139847
\end{verbatim}

%----------------------------------------------------------------------------------------
\newpage
\section{Metodou nejmenších čtverců hledejte aproximační polynomy pro~určené stupně.}

\par Zvoleno zadání a) $n=2$

\begin{center}
	\begin{tabular}{|c|c|c|c|c|c|}
		\hline
		$x$ & 1 & 1,5 & 2 & 2,5 & 3 \\
		\hline
		$f(x)$ & 0,837 & 0,192 & -0,950 & -1,095 & 1,344 \\
		\hline
	\end{tabular}
\end{center}

\par ~\\

%%%% Výpočet
\par Hledaný polynom bude ve~tvaru $P_2(x)=a_0+a_1x+a_2x^2$ a~odpovídajcí systém
normálních rovnic v~zápisu skalárního součinu

\begin{equation}
\arraycolsep=1.4pt\def\arraystretch{1.8}
\begin{array}{lclclcl}
\label{eq:norm}
(y,1) & = & a_0(1,1) & + & a_1(x,1) & + & a_2(x^2,1) \\
(y,x) & = & a_0(1,x) & + & a_1(x,x) & + & a_2(x^2,x) \\
(y,x^2) & = & a_0(1,x^2) & + & a_1(x,x^2) & + & a_2(x^2,x^2) \\
\end{array}
\end{equation}

bude nabývat hodnot z~tabulky:

\[\arraycolsep=1.4pt\def\arraystretch{1.8}
\begin{array}{lclcl}
(y,1) & = & 0,837 + 0,192 - 0,950 - 1,095 + 1,344 & = & 0,328 \\
(y,x) & = & 0,837 +  0,288 - 1,9 - 2,7375 + 4,032 & = & 0,5195 \\
(y,x^2) & = & 0,837 + 0,432 - 3,8 - 6,84375 + 12,096 & = & 2,72125 \\
(1,1) & = & 1 + 1 + 1 + 1 + 1 & = & 5 \\
(x,1) & = & (1,x) = 1 + 1,5 + 2 + 2,5 + 3 & = & 10 \\
(x^2,1) & = & (1,x^2) = (x,x) = 1 + 2,25 + 4 + 6,25 + 9 & = & 22,5 \\
(x,x^2) & = & (x^2,x) = 1 + 3,375 + 8 + 15,625 + 27 & = & 55 \\
(x^2,x^2) & = & 1 + 5,0625 + 16 + 39,0625 + 81 & = & 142,125 \\
\end{array}
\]

\par Výsledná soustava normálních rovnic tedy bude mít po~dosazení do~(\ref{eq:norm})
následující podobu:

\begin{equation}
\label{eq:lin}
\arraycolsep=1.4pt\def\arraystretch{1.8}
\begin{array}{lclclcl}
0,328 & = & a_0\cdot5 & + & a_1\cdot10 & + & a_2\cdot22,5 \\
0,5195 & = & a_0\cdot10 & + & a_1\cdot22,5 & + & a_2\cdot55 \\
2,72125 & = & a_0\cdot22,5 & + & a_1\cdot55 & + & a_2\cdot142,125 \\
\end{array}
\end{equation}

\newpage
\par Soustavu lienárních algebraických rovnic (\ref{eq:lin}) lze vyřešit například užitím
Cramerova pravidla. Uvažujme následující funkci v~prostředí \texttt{MATLAB}, která
realizuje výpočet neznámých podle Cramerova pravidla a~pracuje s~předanou maticí $A$
a~sloupcovým vektorem pravých stran $b$.

\begin{verbatim}
function x = cramerSolve(A, b)
[m n] = size(b);
v = zeros(m,1);
Ai = A;

for k=1:m
    Ai(:,k) = b;
    v(k) = det(Ai);
    Ai(:,k) = A(:,k);
end

detA = det(A);
x = v / detA;
\end{verbatim}

\par ~\\

\par A~její následné zavolání s~odpovídajícími argumenty:\\

\begin{verbatim}
>> A = [ 5 10 22.5; 10 22.5 55; 22.5 55 142.125 ]
>> b = [ 0.328 0.5195 2.72125 ]'
>> a = cramerSolve(A, b)
a =

    7.3398
   -8.2432
    2.0471
\end{verbatim}

\par Hledaný polynom je tedy \doubleunderline{$P_2=7,3398 -8,2432x + 2,0471x^2$}. \\

\par ~ \\
\par Ověření funkcí \texttt{polynomFit2.m}
z~\path{N:\UKAZKY\Prazak\NUMA\B_cviceni\cv06\matlab}:

\begin{verbatim}
>> x = [ 1 1.5 2 2.5 3 ];
>> y = [ 0.837 0.192 -0.95 -1.095 1.344 ];
>> ap = PolynomFit2(x, y, 2)
ap =

    7.3398
   -8.2432
    2.0471
\end{verbatim}

%----------------------------------------------------------------------------------------
% KONEC
%----------------------------------------------------------------------------------------

\end{document}
