%----------------------------------------------------------------------------------------
%	PACKAGES AND OTHER DOCUMENT CONFIGURATIONS
%----------------------------------------------------------------------------------------

\documentclass[a4paper,10pt]{article} 

\usepackage[czech]{babel} % čeština
\usepackage{tabularx}

\usepackage{fontspec} % fonty
\defaultfontfeatures{Mapping=tex-text}
\setmainfont{Calibri-Light} % Main document font
%%\setmainfont{Lucida Grande}

\usepackage{xunicode,xltxtra,url,parskip} % Formatting packages

\usepackage[usenames,dvipsnames]{xcolor} % Required for specifying custom colors

\usepackage[big]{layaureo} % Margin formatting of the A4 page, an alternative to layaureo
                           % can be \usepackage{fullpage} To reduce the height of the top
                           % margin uncomment: \addtolength{\voffset}{-1.3cm}

\usepackage{hyperref} % Required for adding links	and customizing them
\definecolor{linkcolour}{rgb}{0,0.2,0.3} % Link color
\hypersetup{colorlinks,breaklinks,urlcolor=linkcolour,linkcolor=linkcolour} % Set link colors throughout the document

\usepackage{titlesec} % Used to customize the \section command
\titleformat{\section}{\Large\scshape\raggedright}{}{0em}{}[\titlerule] % Text formatting of sections
\titlespacing{\section}{0pt}{3pt}{3pt} % Spacing around sections

%----------------------------------------------------------------------------------------
%	PDF settings
%----------------------------------------------------------------------------------------

\author{Bc. Jan Hladěna}
\hypersetup
{
    pdfauthor={Bc. Jan Hladěna},
    pdfsubject={KF DORDB - Semestrální projekt},
    pdftitle={DORDB Semestrální projekt}
}

\begin{document}
\pagestyle{empty} % Removes page numbering

%\font\fb=''[cmr10]'' % Change the font of the \LaTeX command under the skills section

%----------------------------------------------------------------------------------------
%	HEADER
%----------------------------------------------------------------------------------------

\begin{minipage}[t]{0.6\textwidth}
\par{ {{K-AI2-1 DORDB Semestrální projekt}}}
\end{minipage}
\begin{minipage}[t]{0.4\textwidth}
  \begin{flushright}
  	\par{ {\textsc{Jan Hladěna \\ \texttt{hladeja1} - \texttt{I1500705} }}}
  \end{flushright}
\end{minipage}

\begin{center}
	\noindent\rule{\textwidth}{0.3pt}
\end{center}

%----------------------------------------------------------------------------------------
%       TEXT
%----------------------------------------------------------------------------------------

\par Tato práce vychází ze~zadání semestrálního projektu zpracovávaného v~předchozím
bakalářském studiu pro~předmět DBS. Jednalo se o~školní zadání na~téma \emph{Řetězec
kaváren}, původní scénář je zde přiložen. Tento projekt byl úspěšně obhájen v~LS
2013/2014. \\

\bigskip

\par Přiložené soubory \\
Původní scénář projektu - \texttt{modely/DBS-scenar\_retezec-kavaren.pdf} \\
ER diagram - \texttt{modely/ERD-hladena\_kavarny.pdf} \\
DDL diagram - \texttt{modely/DDL-hladena\_kavarny.pdf} \\

\par Skript vytvářející základní schéma databáze - \texttt{sql-skripty/01-DDL.sql}.
Oproti původnímu zadání je úvazek v~tabulce \texttt{PracovniPomer} realizován absolutním
číslem udávajícím počet hodin týdně, nikoliv relativním poměrem vyjádřeným procenty.

\par Skript naplňující databázi testovacími daty - \texttt{sql-skripty/03a-populate.sql}.

%%%%%%%%%%%%%%%%%%%%%%%%%%%%%%%%%%%%%%%%%%%%%%%%%%%%%%%%%%%%%%%%%%%%%%%%%%%%%%%%%%%%%%%%%%

\section{Popis integritních omezení}

\subsection*{Triviální integritní omezení}

\par Skript vytvářející triviální integritní
omezení - \texttt{sql-skripty/02-Trivialni-IO.sql}.

\par Skript porušující integritní omezení - \texttt{sql-skripty/07-Poruseni-IO.sql}.
\par Výstup pro skript porušující integritní
omezení - \texttt{sql-vystupy/07-Poruseni-IO.txt}.

\bigskip

\par Organizační jednotka - tabulka \texttt{OrgJednotka}

\begin{itemize}
	\item{ID organizační jednotky musí být v~celém systému jedinečné}
	\item{ID organizační jednotky musí být čtyřmístné kladné číslo}
\end{itemize}

\par Pracovní poměr - tabulka \texttt{PracovniPomer} 

\begin{itemize}
	\item{Každý úvazek musí být kladný a~nesmí překročit 40~hodin}
	\item{Každý pracovní poměr nesmí končit dříve, než začne}
\end{itemize}

\par Produkt - tabulka \texttt{Produkt} 

\begin{itemize}
	\item{Cena produktu musí být kladným číslem}
\end{itemize}


\subsection*{Dodatečná integritní omezení}

\par Skript vytvářející dodatečná integritní
omezení - \texttt{sql-skripty/03-Dodatecna-IO.sql}.

\bigskip

%%\par Pracovní poměr - tabulka \texttt{PracovniPomer} 

\begin{itemize}
	\item{Organizační jednotka je v~každé zemi jen jedna}
	\item{Všechny obchody spadají pod~hlavní organizační jednotku příslušnou dané zemi,
	kde se obchod nachází}
	\item{Celkový úvazek každého zaměstnance v~každém z~překrývajících se období nesmí
	překročit 40~hodin týdně}
	\item{Pracovník, který je manažerem některé z~organizačních jednotek, nesmí mít
	v~aktuálním období celkový úvazek na~jiných pracovních pozicích vyšší než 20~hodin
	týdně}
\end{itemize}

%%%%%%%%%%%%%%%%%%%%%%%%%%%%%%%%%%%%%%%%%%%%%%%%%%%%%%%%%%%%%%%%%%%%%%%%%%%%%%%%%%%%%%%%%%

\section{Návrh API \uv{business logiky}}

\begin{enumerate}
\item{\emph{Změna oddělení zaměstnance ze~starého na~nové.} Procedura na~základě předaných
argumentů ukončí pracovní poměr ve~starém oddělení a~přesune celý úvazek daného
zaměstnance do~nového oddělení. \\
\emph{zmena\_odd(id zaměstnance, id starého oddělení, id nového oddělení);}\\
%%\verb|zmena_odd(zam IN NUMBER, old_odd IN NUMBER, new_odd IN NUMBER);|
}

\item{\emph{Povýšení zaměstnance na~manažera některé organizační jednotky.} Procedura
na~základě předaných argumentů zjistí celkový současný úvazek daného zaměstnance, který,
překračuje-li 20~hodin, bude poměrově přerozdělen do~stávajících pracovních poměrů tak,
aby nebylo celkově překročeno 20~hodin. Přerozdělení bude realizováno ukončením
stávajících pracovních poměrů a~vytvořením nových. Zaměstnanec bude následeně jmenován
manažerem dané organizační jednotky. \\
\emph{novy\_manager(id zaměstnance, id organizační jednotky);}\\
}

\end{enumerate}

\par Skript vytvářející balíček a~business logiku - \texttt{sql-skripty/04-API-package.sql}.
\par Skript ověřující business logiku - \texttt{sql-skripty/05-API-verify.sql}.
\par Výstup ověření business logiky - \texttt{sql-vystupy/05-API-verify.txt}.

%%%%%%%%%%%%%%%%%%%%%%%%%%%%%%%%%%%%%%%%%%%%%%%%%%%%%%%%%%%%%%%%%%%%%%%%%%%%%%%%%%%%%%%%%%

\section{Definované dotazy nad~schématem}

\begin{enumerate}
	\item{Nejstarší zaměstnanci - vypsání seznamu tří nejstarších zaměstnanců firmy včetně
	data jejich nástupu a~oddělení.}
	\item{Letištní sortiment - výpis produktů, které jsou k~dostání na~letištích.}
	\item{Čajovny - seznam adres všech čajoven na~světě.}
	\item{Bývalí zaměstnanci - seznam adres bývalých zaměstnanců.}
	\item{Management OU - seznam manažerů všech organizačních jednotek.}
\end{enumerate}

\par Skript vytvářející definované dotazy - \texttt{sql-skripty/06-Dotazy.sql}.
\par Výstupy dotazů - \texttt{sql-vystupy/06-Dotazy.txt}.

%%%%%%%%%%%%%%%%%%%%%%%%%%%%%%%%%%%%%%%%%%%%%%%%%%%%%%%%%%%%%%%%%%%%%%%%%%%%%%%%%%%%%%%%%%

\section{Odvozené hodnoty}

\par Tabulka \texttt{Zamestnanec} bude doplněna o~sloupec \texttt{celk\_uvazek},
obsahující informaci o~celkovém aktuálním úvazku daného zaměstnance. \\ 

\par Skript doplňující sloupec s~odvozenou hodnotou - \texttt{sql-skripty/08-Odvozene.sql}.

%%%%%%%%%%%%%%%%%%%%%%%%%%%%%%%%%%%%%%%%%%%%%%%%%%%%%%%%%%%%%%%%%%%%%%%%%%%%%%%%%%%%%%%%%%

\section{Schéma v~objektové podobě}

\par Skript vytvářející schéma v~objektové
podobě a~importující neobjektová data - \texttt{sql-skripty/09-Objektove-schema.sql}.

\par Skript provádějící dotazy nad~objektovým
schématem - \texttt{sql-skripty/10-Objektove-dotazy.sql}.

\par Výstup dotazů provedených nad~objektovým
schématem - \texttt{sql-vystupy/10-Objektove-dotazy.txt}.

%%%%%%%%%%%%%%%%%%%%%%%%%%%%%%%%%%%%%%%%%%%%%%%%%%%%%%%%%%%%%%%%%%%%%%%%%%%%%%%%%%%%%%%%%%

\section{Slovní návrh faktů a~dimenzí pro~datový sklad}

\par Tabulka faktů - \emph{pracovní poměr}, dimenze \emph{zaměstnanec}, \emph{oddělení},
\emph{datum}, \emph{úvazek}. Možnosti analýzy přestupů zaměstnanců mezi odděleními,
na které, oddělení v~jaké době pracovalo nejvíce zaměstnanců, kdo měl po~jakou dobu jak
velký úvazek.\\

%%%%%%%%%%%%%%%%%%%%%%%%%%%%%%%%%%%%%%%%%%%%%%%%%%%%%%%%%%%%%%%%%%%%%%%%%%%%%%%%%%%%%%%%%%


\end{document}
